\documentclass{article}

\usepackage{hyperref,xspace}

\oddsidemargin 0cm
\textwidth 17cm


\newcommand\eg{{\em e.g.}\xspace}
\newcommand\ie{{\em i.e.}\xspace}

\begin{document}


\title{\large\bf CS 3100 -- Models of Computation -- Fall 2011 \\
 This assignment is worth 8\% of the total points for assignments \\
 100 points total}

\maketitle

\begin{center}
{\bf Assignment 3, Posted on: 9/6  Due: 9/15 {\bf Thursday} 11:59pm}
\end{center}

\thispagestyle{empty}

\begin{enumerate}

\item ({\bf 20 points})\label{q1}
  Write a Python function \verb|recognizes(D, N)| that returns
  all strings of length $0 \leq i \leq N$  recognized by the given DFA {\tt D}.
  Assume that $N \geq 0$.
  %
  Test it out on the the DFA that recognizes all strings ending in 
  $0101$ that you constructed in Assignment 2 for $N=5$.
  %
  Submit the function in a file {\tt recognizes.py} as well as
  an ASCII record of your testing session 
  as file \verb|recognizes_tests.out|.

\item ({\bf 40 points}) \label{q2}
 Define a DFA that accepts all strings over \{0,1\} 
  such that every block of four consecutive positions contains at 
  least two 0s.
  %
  (This means: {\bf If} there are four consecutive positions, {\bf Then} in those
  four positions, there must be at least two 0s.)
  %
  Call this language $L_{00}$.
  %
  Build this DFA using the \verb|mk_dfa| call (we will supply you
  a working \verb|mk_dfa| for this assignment).
  %
  Next, use \verb|dot_dfa| and print this DFA out.
  %
  Submit the PDF drawing of this DFA, as file {\tt L00.pdf}.
  %
  Test this DFA on 12 strings
  including two (2) strings of length $< 5$, 
  five (5) strings that are accepted and of length $\ge 6$
and
  five (5) strings that are rejected and of length $\ge 6$.
  %
  Submit an ASCII record of your testing session
  as file \verb|L00_tests.out|.

\item ({\bf 20 points}) Draw a DFA for Question 3 of {\tt notes5.pdf}.
  Next, enter this DFA and generate a PDF drawing for it. Argue why
  this DFA works (in about 3-4 sentences), and also use function
  \verb|accepts| to demonstrate that indeed it works on five (5) 
  strings in the language and five (5) strings not in the language.
  Submit your PDF as \verb|notes5_qn3_DFA.pdf|
  and your writeup as \verb|notes5_qn3_DFA.out|.
  
\item ({\bf 20 points}) Draw a DFA for Question 5 of {\tt notes5.pdf}.
  Next, enter this DFA and generate a PDF drawing for it, and submit it. Argue why
  this DFA works (in about 3-4 sentences), and also use function
  \verb|accepts| to demonstrate that indeed it works on five (5) 
  strings in the language and five (5) strings not in the language.
  Submit your PDF as \verb|notes5_qn5_DFA.pdf|
  and your writeup as \verb|notes5_qn5_DFA.out|.

\end{enumerate}

\end{document}



